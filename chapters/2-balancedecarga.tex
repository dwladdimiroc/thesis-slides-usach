\section{Balance de carga}
\subsection*{Perspectivas de balance de carga}
\begin{frame}{Balance de carga}{Perspectivas de balance de carga}
\begin{itemize}
\item Perspectivas al problema de balance de carga en procesamiento de \textsl{stream}
\begin{itemize}
	\item Recursos físicos
	\item Grafo de operadores
\end{itemize}
\item Para la optimización del sistema, se presentan dos enfoques \cite{Dong06schedulingalgorithms}
\begin{itemize}
	\item Estático
	\item Dinámico
\end{itemize}
\end{itemize}
\end{frame}

\begin{frame}{Estado del arte}{Enfoque dinámico}
\begin{itemize}
\item En el estado del sistema
\item Las variables y estados de cada uno de sus atributos
\item Cambios en el sistema ante una anomalía
\item Tipos de modelo para las soluciones:
\begin{itemize}
	\item Reactivo \cite{GulisanoJPSV12}
	\item Predictivo \cite{NguyenSGSW13}
\end{itemize}
\end{itemize}
\end{frame}

\subsection*{Técnicas de balance de carga}
\begin{frame}{Balance de carga}{Técnicas de balance de carga}
\begin{itemize}
\item Existes distintas técnicas que son utilizadas en ambos modelos
\item Por ejemplo:
\begin{itemize}
	\item Planificación determinista \cite{XuCTS14, DongTS07}
	\item Descarte de eventos \cite{SheuC09}
	\item Migración \cite{XingZH05}
	\item Fisión \cite{GulisanoJPSV12, IshiiS11, GedikSHW14, FernandezMKP13}
\end{itemize}
\end{itemize}
\end{frame}